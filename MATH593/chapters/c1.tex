\section{Ring homomorphism, Quotient Ring}

\begin{definition}[Ring Homomorphism]
    Let $X, Y$ be rings. A \textbf{Ring Homomorphism} is a map $f: X\to Y$ satisfying the following properties:
    \begin{itemize}
        \item $f(1) = 1$.
        \item $\forall x_1, x_2\in X, f(x_1) + f(x_2) = f(x_1 + x_2)$.
        \item $\forall x_1, x_2\in X, f(x_1 x_2) = f(x_1) f(x_2)$
    \end{itemize}
\end{definition}

\begin{definition}[Quotient Ring]
    Let $R$ be a ring and $I \subseteq R$ a two-sided ideal. The \textbf{Quotient Ring} $(R/I)$ is defined as $(R/\sim)$ with an equivalence relation $\sim$ where $a\sim b$ if and only if $a - b = I$. Elements in $(R/I)$ are denoted as $\bar{a}$, where $\bar{a} = \bar{b}$ if and only if $a\sim b$.
\end{definition}

The natural homomorphism $\pi_I: R\to (R/I)$ is defined as $\pi(a) = \bar{a}$, which satisfies the \emph{universal property of quotient rings}:

\begin{theorem}[Fundamental Theorem of Ring Homomorphisms]
    Let $\varphi: R\to S$ be a ring homomorphism, $I$ a two-sided ideal s.t. $I\subseteq \ker \varphi$, and $\pi$ be the natural ring homomorphism from $R$ to $(R/I)$. Then there exists a unique ring homomorphism $f: R/I \to S$ s.t. the following diagram commutes,
    \begin{figure}[htbp]
        \centering    
        \begin{tikzcd}[]
            R \arrow[rrdd, two heads, "\pi"] \arrow[rr, "\varphi"] & & S \\
            & & \\
            & & R/I \arrow[uu, "f"'] 
        \end{tikzcd}
    \end{figure}
    i.e. $\varphi = f\circ \pi$.
\end{theorem}

\begin{proof}
    It suffices to prove that $f$ exists and is unique, and verify that $f$ is indeed a ring homomorphism.
    \begin{itemize}
        \item \textbf{Uniqueness.} By the requirement that $f$ should make the diagram commute, $f(\bar{a}) = \varphi(a),\ \forall a\in R$. Uniqueness of $f$ follows from the fact that $\varphi$ maps every element in $R$ to a unique element in $S$.
        \item \textbf{Existence.} It suffices to verify that $f$ is well-defined, i.e. does not vary w.r.t. change of representative in $(R/I)$. For all $a, b\in R$ s.t. $\bar{a} = \bar{b}$, $(a - b)\in I \implies \varphi(a - b) = 0 \implies \varphi(a) = \varphi(b)$ since $\varphi$ is a ring homomorphism. By the uniqueness of $f$ it is specified that $f(\bar{a}) = \varphi(a)$, which implies that for all $\bar{a} = \bar{b}\in (R/I), f(\bar{a}) = \varphi(a) = \varphi(b) = f(\bar{b})$.
        \item \textbf{$f$ is indeed a homomorphism.} This follows from the fact that $\varphi$ is a ring homomorphism.
    \end{itemize}
\end{proof}

\section{Ring of Fractions}

\begin{definition}[Multiplicative System]
    A subset $S\subseteq R$ for a ring $R$ is a \textbf{multiplicative system} if $1\in S$, and $\forall s_1, s_2\in S, s_1\cdot s_2\in S$, where $\cdot$ is the multiplication in $R$.
\end{definition}

\begin{definition}[Ring of Fractions]
    Let $R$ be a commutative ring, with $S\subseteq R$ a multiplicative subset, the \textbf{ring of fraction} $S^{-1}R$ is defined as $R\times S / \sim$, where $(s_1, r_1) \sim (s_2, r_2)$ if and only if there exists $t\in R$ s.t. $t(s_1 r_2 - s_2 r_1) = 0$. $(s, r) \in S^{-1}R$ is denoted as $\frac{s}{r}$. The definition of operations follows directly from analogy of that in $\Q$.

    The natural homomorphism (inclusion map) from $R$ to $S^{-1}R$ is defined as $r \hookrightarrow \frac{r}{1}$.
\end{definition}

\begin{remark}\upshape
    If $R$ is an integral domain, then $(s_1, r_1) \sim (s_2, r_2)$ iff $s_1 r_2 = s_2 r_1$, as for $\Q$.
\end{remark}

\begin{remark}
    If $R$ is not an integral domain, and $S$ contains zero divisors, then the inclusion map ceases to be injective, as choosing $t$ s.t. it satisfies $ts_1 = ts_2 = 0$ for some $s_1, s_2$ that are zero divisors gives $\varphi(s_1) = \varphi(s_2)$. Changing $R$ to an integral domain guarantees that the inclusion map $\varphi$ is injective.
\end{remark}

\begin{proposition}
    $\sim$ is an equivalence relation.
\end{proposition}

\begin{proof}
    It is clear that $\sim$ is reflexive and symmetric. For transitivity, consider $(s_1, r_1)\sim (s_2, r_2) \wedge (s_2, r_2)\sim (s_3, r_3)$. That is, there exists some $t_1, t_2\in R$ s.t. 
    \[
        \begin{cases}
            t_1 (s_1 r_2 - s_2 r_1) = 0 \\ t_2 (s_2 r_3 - s_3 r_2) = 0 
        \end{cases}
        \ \implies\ t_1 t_2 (s_1 r_2 s_3 - s_2 r_1 s_3) = t_1 t_2 (s_1 s_2 r_3 - s_2 r_1 s_3) = t_1 t_2 s_2 (s_1 r_3 - s_3 r_1) = 0
    \]
\end{proof}

\begin{remark}
    Notice that if $s\in S, then \frac{s}{a}$ for $a\in R$ is invertible. This tends more to a field, with more elements being ``reachable'' via multiplying an element from one side. A direct consequence is that less ideals exist in $S^{-1}R$, with ideals in $R$ whose generators differ by a factor that divides $s$ being identified in $S^{-1}R$. 
\end{remark}

\begin{remark}
    It is required that $R$ is commutative is to preserve the most structures from $R$, i.e. ensure that $S^{-1}I$ is an ideal for all ideals in $R$. This is due to the addition in action:
    \[
        \forall\ \frac{r_1}{s_1}, \frac{r_2}{s_2} \in S^{-1}R, \qquad \frac{r_1}{s_1} + \frac{r_2}{s_2} = \frac{r_1 s_2 + s_1 r_2}{s_1 s_2} 
    \]
    which indicates that $S^{-1}I$ is a two-sided ideal if and only if $I\subseteq R$ is a two-sided ideal. For one-sided (left/right) ideal the property is not fully inherited. 
\end{remark}

\begin{theorem}[Universal Property of Ring of Fractions]
    Suppose $R$ and $T$ are commutative rings, with $\varphi$ the inclusion of $R$ into $S^{-1}R$. Then for $f: R\to T$ s.t. $\forall s\in S, f(s)$ is invertible in $T$, there exists a unique ring homomorphism $g$ s.t. $f = g\circ \varphi$, i.e. make the following diagram commute:
    \begin{figure}[htbp]
        \centering    
        \begin{tikzcd}[]
            R \arrow[rrdd, "f"] \arrow[rr, hookrightarrow, "\varphi"] & & S^{-1}R \arrow[dd, "g"]  \\
            & & \\
            & & T
        \end{tikzcd}
    \end{figure}
\end{theorem}

\begin{proof}
    Adopt the same strategy as in the previous section: 
    \begin{itemize}
        \item \textbf{Existence.} For all $\frac{a}{s}\in S^{-1}R$, $g(\frac{a}{s}) := f(a) (f(s))^{-1}$ which is well-defined since $f$ is required to map all elements in $S$ to invertible elements. $g$ being a ring homomorphism follows from the fact that $f$ is a ring homomorphism. 
        \item \textbf{Uniqueness.} Follows from specifying $g(\frac{a}{s}) := f(a) (f(s))^{-1}$.
    \end{itemize}
\end{proof}

\begin{remark}
    If $S := R\smallsetminus\{0\}$, then $S^{-1}R$ is the whole field, with localization equivalent to completion of inverse of $R$.
\end{remark}

\begin{definition}
    A commutative ring $R\neq \{0\}$ is \textbf{local} if it admits a unique maximal ideal $M$. Local rings are denoted by a pair $(R, M)$.
\end{definition}

\begin{example}
    Let $R$ be a commutative ring, with $\mathfrak{p} \subseteq R$ a prime ideal. Let $S = R\smallsetminus p$ be a multiplicative system. Then the ring $S^{-1}R$ is local, with the maximal ideal of it being $S^{-1}\mathfrak{p}$. This results from the fact that $S^{-1}I$ is an ideal if and only if $I$ is an ideal in $R$. Further since $\Z$ is a PID (see next section), all prime ideals are maximal, $S^{-1}\mathfrak{p}$ is indeed maximal. The fact that there is only one such maximal ideal results from that all other primes are in $S$, i.e. $S^{-1}\mathfrak{p}'=S^{-1}R$ for all $\mathfrak{p}' \neq \mathfrak{p}$. 
\end{example}

\begin{proposition}
    Let $R\neq \{0\}$ be a commutative ring. Then $R$ being local if and only if for all $a\in R$, either $a$ is invertible or $(1-a)$ is invertible. In this case, the maximal ideal $M$ is the set of all non-invertible elements. 
\end{proposition}

\begin{proof}
    Proceed by showing implication in both directions:
    \begin{itemize}
        \item[$\Rightarrow$:] Suppose that $(R, M)$ is the local ring of interest. Proceed by showing a contradiction: suppose that both $a$ and $(1-a)$ are non-invertible. Then since $R$ is local $(a) \subseteq M$, $(1-a) \subseteq M$ indicating that $1\in M$ which is a contradiction. In this case for all $a$ non-invertible, $(a) \subseteq M$, which implies that $M$ is the set of all non-invertible elements. 
        \item[$\Leftarrow$:] Define set $M := \{ a\in R \mid \forall x\in R, ax \neq 1 \}$. By construction if $M$ is an ideal then it must be maximal, as including an invertible element expands the ideal to the whole ring. Verify that $M$ is indeed an ideal:
            \begin{itemize}
                \item \emph{Closed with addition}. Proceed via showing that the contraposition. Suppose that there exists $a, b\in R$ s.t. both $a$ and $b$ are non-invertible, but there exists some $c\in R$ s.t. $c(a + b) = 1$. Then $ca = 1 - (cb)$ is non-invertible, which implies that $1 - ca$ is invertible. But notice $1 - ca = cb$ is also non-invertible, which is a contradiction.
                \item \emph{Absorption with multiplication.} This simply results from the fact that a non-invertible element multiplied by a unit is still non-invertible. 
            \end{itemize}

        Further notice that this is indeed the only maximal ideal, as for all $u\in R\smallsetminus M$, it is invertible, i.e. for all ideals $I\subseteq R$, $u\in I\implies 1\in I \implies I = R$. Therefore $(R, M)$ is local. 
    \end{itemize}
\end{proof}

\section{Polynomial Rings}

\begin{definition}[R-algebra]
    Let $R$ be a ring. Then a ring $S$ is an \textbf{$R$-algebra} for the specific $R$ mentioned if there exists a ring homomorphism $\varphi: R\to S$ s.t. $\forall r\in R, s\in S, \varphi(r)s = s\varphi(r)$. When the homomorphism needs to be specified, the algebra is often denoted as a pair $\pair{S, \varphi}$
\end{definition}

\begin{remark}
    An $R$-algebra is a two-sided $R$-module, which can be regarded as a generalization of the structure in $R$. $R$ itself is not necessarily commutative, which implies that the associated homomorphism maps $R$ to the center of $S$.
\end{remark}

\begin{definition}[Morphism of $R$-algebras]
    Let $\pair{R_1, f_1}, \pair{R_2, f_2}$ be $R$-algebras. A \textbf{Morphism of $R$-algebras} is a ring homomorphism $\varphi: R_1 \to R_2$ s.t. the following diagram commute; i.e. $f_2 = \varphi \circ f_1$:
    \begin{figure}[htbp]
        \centering    
        \begin{tikzcd}[]
            R \arrow[rrdd, "f_2"] \arrow[rr, "f_1"] & & R_1 \arrow[dd, "\varphi"]  \\
            & & \\
            & & R_2
        \end{tikzcd}
    \end{figure}
\end{definition}

\begin{definition}[$R$-subalgebra]
    Let $\pair{S, f_s}$ be a $R$-algebra for $R$ a ring. $\pair{T, f_t}$ is a \textbf{$R$-subalgebra} of $S$ if $T$ is a $R$-algebra, with $f_t(R) \subseteq S$ ; and there exists a morphism $\varphi$ from $T$ to $S$, i.e. $\varphi$ makes the following diagram commute:
    \begin{figure}[htbp]
        \centering    
        \begin{tikzcd}[]
            R \arrow[rrdd, "f_s"] \arrow[rr, "f_t"] & & T \arrow[dd, hookrightarrow, "\varphi"]  \\
            & & \\
            & & S
        \end{tikzcd}
    \end{figure}
\end{definition}

\begin{definition}[Polynomial Ring]
    Let $R$ be a commutative ring. The \textbf{polynomial ring of $R$}, denoted $R[x]$, is defined as
    $$
        R[x] := \left\{ \sum\limits_{i=0}^{n} c_i x^i \mid n\in\N, c_i\in R \right\}
    $$
    with the addition and multiplication the same as in polynomials over $\Z$. The natural inclusion from $R$ to $R[x]$ is defined as $r\mapsto r$ which is a polynomial of degree 0.
\end{definition}

\begin{remark}
    If $R$ is a domain, then $R[x]$ is also a domain (consider the product of terms with highest degree); where $\deg (f g) \leq \deg (f) + \deg (g)$.
\end{remark}

\begin{theorem}[Universal Property of Polynomial Ring]
    Let $R$ be a ring and $\pair{S, f}$ an $R$-algebra, and $\varphi$ be the inclusion map from $R$ to $R[x]$. For all $s\in S$, there exists a unique morphism of $R$-algebra $g: R[x] \to S$ s.t. $g(x) = a$, and the following diagram commutes, i.e. $f = g\circ \varphi$:
    \begin{figure}[htbp]
        \centering    
        \begin{tikzcd}[]
            R \arrow[rrdd, hookrightarrow, "f"] \arrow[rr, "\varphi"] & & R[x] \arrow[dd, "g"]  \\
            & & \\
            & & S
        \end{tikzcd}
    \end{figure}
\end{theorem}

\begin{proof}
    Proceed similarly by first determining the form that $g$ takes, and then showing the uniqueness and existence. 
    \begin{itemize}
        \item \textbf{Uniqueness.} Since it is required that $g$ is a morphism of $R$-algebras, we have
        \[
            g\left( \sum\limits_{i=0}^n a_i x^i\right) = \sum\limits_{i=0}^{n} g(a_i) g(x^i) = \sum\limits_{i=0}^{n} f(a_i) g(x^i) = \sum\limits_{i=0}^{n} f(a_i) a^i
        \]
        by the requirement that $g(x) = a$. This is the only form that $g$ could take, and thus proves its uniqueness.
        \item \textbf{Existence.} For existence it suffices to check that $g$ is indeed a ring homomorphism. By the uniqueness $g$ is fixed by sending $x\in R[x]$ to $a\in R$. Notice that $R$ is commutative, which indicates that both left and right composition is satisfied; with the addition condition verified in the uniqueness part. 
    \end{itemize}
\end{proof}

\begin{theorem}[Universal Property of Polynomial Ring of Several Variables]
    Let $A$ be a commutative $R$-algebra and $g$ be the inclusion map from $R$ to $R[x_1, \cdots, x_n]$ with a fixed $n$. For every $R$-algebra $S$ and $(a_1, \cdots, a_n)\in S$, there exists a unique homomorphism of $R$-algebra $h: R[x_1, \cdots, x_n] \to S$ s.t. $h(x_i) = a_i$ for all $i\in \llbracket 1, n \rrbracket$, and the following diagram commutes, i.e. $f = h\circ g$:
    \begin{figure}[htbp]
        \centering
        \begin{tikzcd}[]
            R \arrow[rrdd, hookrightarrow, "g"] \arrow[rr, "f"] & & S \\
            & & \\
            & & R[x_1, \cdots, x_n] \arrow[uu, "h"]
        \end{tikzcd}
    \end{figure}
\end{theorem}

\begin{proof}[Sketch of Proof]
    The idea is similarly consider substitution $x_i \mapsto a_i$, and proceed to verify that this is indeed a ring homomorphism. One step that requires caution is that polynomials of several variables are defined in an inductive manner; therefore here proof should also be done inductively, on the number of variables involved. 
\end{proof}

Using polynomial of several variables, it is clearer to formalize the ``generating set'' of a ring via specifying which element each variable maps to:

\begin{definition}[Finitely Generated $R$-algebra]
    Let $R$ be a commutative ring, with $A$ a commutative $R$-algebra. Fix $(a_1, \cdots, a_n)\in A$. By the universal property of polynomial of several variables, there exists a unique homomorphism $\varphi: R[x_1, \cdots, x_n]$ s.t. $\varphi(x_i) = a_i$. Then the subalgebra $\im \varphi$ is said to be \textbf{generated} by $\{a_1, \cdots, a_n\}$. 
\end{definition}

\begin{remark}
    Using the samre-formalization as in the definition above, $\im \varphi$ is smallest $R$-subalgebra of $A$ that contains $\{ a_1, \cdots, a_n \}$. 
\end{remark}

\begin{proof}
    It is clear that $\im \varphi$ contains $\{ a_1, \cdots, a_n \}$. To see that it is smallest, suppose there is a smaller one $A'$, then there must be some $\sum_{i=0}^n a_i x^i \notin A'$, which contradicts with the fact that a ring should be closed. 
\end{proof}

Notice that in the definition of polynomial ring it is only required that $x$ could be multiplied with powers of itself. This enables making polynomial a representation of groups:

\begin{definition}[Group Ring]
    Let $R$ a commutative ring, and $G$ a group. A \textbf{group ring of $R$ on $G$} is defined as
    \[
        R[G] := \left\{ \sum\limits_{g\in G} a_g g \mid a_g \in R \right\}
    \]
    with the addition and multiplication the same as that in the polynomial ring. 
\end{definition}

\begin{remark}
    The operation between the ring and the group is not required to be defined and is simply a notation. The polynomial cannot admit any structure that is more complicated (e.g. changing the group to be a ring) as otherwise the addition will not be well-defined. 
\end{remark}

\section{Ideals}

\begin{definition}[Finitely-Generated Ideals]
    Let $R$ be a ring. Then
    \begin{itemize}
        \item Let $(I_{\alpha})$ be a family of ideals for $\alpha\in\Lambda$ the index set, then the \textbf{ideal generated by (sum of)} $(I_{\alpha})$ is defined as
        \[
            \sum\limits_{\alpha\in \Lambda' \subseteq \Lambda} I_{\alpha} := \left\{ \sum\limits_{\alpha\in \Lambda'} a_\alpha \Big| a_{\alpha} \in I_{\alpha}, \abs{\Lambda'} \text{ finite}  \right\}
        \]
        \item Alternatively one could consider the \textbf{ideal generated by (product of)} two ideals (which can be easily extended to several ideal cases) $I$ and $J$ to be
        \[
            I\cdot J := \left\{ \sum\limits_{i=1}^n a_i b_i \Big| n\in\Z_{>0}, a_i \in I, b_i\in J \forall i \right\}
        \]
        \item Suppose further that $R$ is commutative. Let $\Lambda := \{\lambda_1, \cdots, \lambda_n\}$ be a subset of $R$. Then the \textbf{ideal generated by} $\Lambda$ is defined as
        \[
            (\lambda_1, \cdots, \lambda_n) := \left\{ \sum\limits_{k=1}^n r_k \lambda_k \Big| r_k\in R \right\}
        \]
    \end{itemize}
\end{definition}

\vspace{1em}
\begin{remark}
    Ideals generated by only one element is principal. For finitely generated ideals, the ideal generated by a set of elements is the same as the ideal generated by the corresponding principal ideals of the elements. This simply results from the fact that $(a) = \left\{ ra | r\in R \right\}$.
\end{remark}

Specify $R$ to be a commutative ring, with $I \subseteq R$ an ideal of $R$. Consider the following special cases of ideals:

\begin{definition}[Radical Ideal]
    $I \subseteq R$ is a \textbf{radical ideal} if for all $a\in R$, $\exists n\in \Z_{>0}\ a^n\in I \implies a\in I$.
\end{definition}

\begin{definition}[Prime Ideal]
    $I \subseteq R$ is a \textbf{prime ideal} if $I \neq R$, and for all $a, b\in R, ab\in I \implies (a\in I) \vee (b\in I)$.
\end{definition}

\begin{definition}[Maximal Ideal]
    $I \subseteq R$ is a \textbf{maximal ideal} if $I \neq R$; and there is no ideal $J$ in $R$ s.t. $I \subsetneqq J \subsetneqq R$. 
\end{definition}

\begin{remark}
    Recall that $R$ is a domain if and only if for all $a, b\in R$, $ab = 0 \implies a = 0 \vee b = 0$. This implies that for any ring $R$ with $\mathfrak{p}$ a prime ideal in it, $R/\mathfrak{p}$ is a domain. 
\end{remark}

\begin{definition}[Reduced Ring]
    A $R$ is a \textbf{reduced ring} if and only if it does not have any nilpotent elements, i.e. for all $u\in R, u^n = 0 \implies u = 0$ for all $n\in \Z_{>0}$.
\end{definition}

\begin{remark}
    For a commutative ring $R$, $I$ is a radical ideal if and only if $R/I$ is a reduced ring.
\end{remark}

\begin{proposition}
    $I$ is a maximal ideal if and only if $R/I$ is a field. 
\end{proposition}

\begin{proof}
    This fact follows directly from the following simple lemma.
\end{proof}

\begin{lemma}
    $R = K$ is a field if and only if it only has two ideals $(0)$ and $(1)$.
\end{lemma}

\begin{proof}
    Consider in both directions:
    \begin{itemize}
        \item[$\Rightarrow$:] If $K$ is a field, then either there are no invertible elements, which in this case the ideal $I$ can only contain 0 as this is the only non-invertible element in a field; or $1$ and therefore every element is in the ideal, as $\forall g\in I, \exists g^{-1} \in K, gg^{-1} = 1 \in I$.
        \item[$\Leftarrow$:] If a ring $R$ has only two ideals $(0)$ and $(1)$, then for all $0\neq u\in R$ consider $(u)$. By hypothesis $(u) = (1)$, i.e. there exists some $u^{-1}\in R$, which implies that $R$ is actually a field.
    \end{itemize} 
\end{proof}

\begin{proposition}
    An ideal being maximal implies that it is prime; and an ideal being prime implies that it is radical. 
\end{proposition}

\begin{proof} 
   \emph{Maximal ideals are prime}. Suppose that $I \subseteq R$ is maximal but is not prime, i.e. there exists some $a, b\in R$ s.t. $ab\in R, a\notin R, b\notin R$. By hypothesis $I \cup \{a\} = R$., i.e. there exists some $r\in R, t\in I$ s.t. $a + rt = 1$. But then $b = ba + (br)t \in I$ which is a contradiction.

   \emph{Prime ideals are radical.} Consider inductively on $a$ and $a^{n-1}$; apply the definition of prime ideals.
\end{proof}

\begin{example}
    Consider counterexamples of the converse of the proposition above:
    \begin{itemize}
        \item $\Z_N$ for $N$ not a power of prime is radical, but not prime.
        \item A trivial case for an ideal being prime but not maximal is $(0)$, where as long as the ring is not a field, it is maximal.
        \item A more interesting case for an ideal being prime but not maximal is for finitely generated non-PIDs, adding a generator to a prime ideal suffices to create a ``larger'' ideal. Take the example $(x) \subseteq R[x]$ where $R$ is a domain, which is prime as $R[x]/\pair{x} \cong R$ is also a field. But $(x) \subseteq (2, x)$ which is not the whole ring.
    \end{itemize}
\end{example}

\section{Noetherian Ring}

\begin{lemma}[Zorn's Lemma] \label{lem:Zorn}
    Suppose that $(P, \leq)$ is an ordered set s.t. every totally order subset $P_0 \subseteq P$ has an upper bound, then $P$ has a maximal element. 
\end{lemma}

\begin{theorem} \label{thm:maximal ideal inclusion}
    Let $I \subseteq R$ be an ideal of a commutative ring $R$. Then there exists some maximal ideal $M$ s.t. $I \subseteq M$.
\end{theorem}

\begin{proof}
    The proof is simply a re-formalization of Zorn's Lemma (Lemma \ref{lem:Zorn}). 
    
    Consider $P := \{ J \subseteq R \mid J \text{ ideals}, I \subseteq J, J \neq R \}$, with the order of inclusion. Take $P_0 := \{I_{\alpha} \mid \alpha\in \Lambda\} \subseteq P$ to be totally ordered. Then $J := \bigcup_{\alpha} I_{\alpha}$ is also an ideal. Further $1\notin J$, otherwise there will exist some $\alpha\in \Lambda$ s.t. $I_{\alpha} = R$, which contradicts the hypothesis. Therefore $J$ is the upper bound for the family $P_0$. Applying Zorn's Lemma finishes the proof. 
\end{proof}

\begin{definition}[Noetherian Ring]
    A ring $R$ is (left) \textbf{Noetherian} if it satisfies the \underline{Ascending Chain Condition (ACC)}, for (left) ideals, i.e. there is no infinite strictly increasing sequence of (left) ideals:
    \[
        I_1 \subsetneqq I_2 \subsetneqq \cdots
    \]
\end{definition}

\begin{proposition}\label{prop:Noeth f.g.}
    Let $R$ be a ring, then the followings are equivalent:
    \begin{enumerate}
        \item $R$ is (left) Noetherian.
        \item Let $P$ be a family of (left) ideals in $R$, then $P$ has a maximal element.
        \item Every (left) ideal in $R$ is finitely generated.  
    \end{enumerate}
\end{proposition}

\begin{proof}
    \begin{itemize}
        \item (i) being equivalent to (ii) is via simply reformalizing the definition.
        \item (i) implies (iii). Proceed by proving the contraposition. Suppose that there exists an ideal $I_0 \subseteq R$ that is not finitely generated, then there exists an infinite sequence of generators of $I_0$ $(a_i)$, $i\in \mathcal{I}$. Then there exists an infinite ACC $(a_1) \subsetneqq (a_1, a_2) \subsetneqq (a_1, \cdots, a_k), \subsetneqq\cdots$. 
        \item (iii) implies (i). Prove by showing a contradiction. Suppose that there exists an infinite ACC $I_1 \subsetneqq I_2 \subsetneqq \cdots \subsetneqq I_k \subsetneqq\cdots$. Then consider $I := \bigcup_{n\geq 1} I_n$. By the hypothesis it is finitely generated, i.e. there exists some $(a_1, \cdots, a_m)$ s.t. $a_i \in I_{n_i}$ for all $i\in \llbracket 1, m \rrbracket$. Define $n := \max\{ n_i \mid i\in \llbracket 1, m \rrbracket \}$. Then $I_n = I_{n+1}$ which is a contradiction.
    \end{itemize}
\end{proof}

\begin{theorem}[Hilbert's Basis Theorem]
    Let $R$ be a commutative Noetherian ring. Then $R[x]$ is a Noetherian ring.
\end{theorem}

\begin{proof}
    By proposition \ref{prop:Noeth f.g.} it suffices to show that every ideal of $R[x]$ is finitely generated. 

    In the case that $I = (0)$, it is finitely generated as $R$ is Noetherian. For the case of that $I \neq (0)$, consider a family of ideals where $f_1 \in I\smallsetminus\{0\}$, with $f_k \in I \smallsetminus (f_1, \cdots, f_k)$ for $k > 1$ s.t. $\deg f_k = \min \{\deg f \mid f \in I \smallsetminus (f_1, \cdots, f_k) \}$. If there exists some $k$ s.t. $(f_1, \cdots, f_k) = I$ then $R[x]$ is by definition Noetherian. Suppose that it is not. Then there exists an infinite ascending chain. Denote $f_n = a_n x^{d_n} + \sum\limits_{k=0}^{d_n - 1} a_k x^k$. From the construction it is clear that $d_1 \leq d_2 \leq\cdots\leq d_n\leq\cdots$.

    Define $I := (a_1, \cdots, a_n \mid n \geq 1)$. By hypothesis $I \subseteq R$, which implies that it is finitely generated. Then there exists some $k$ s.t. $I = (a_1, \cdots, a_k)$, with $d_i \geq 1$ (otherwise suppose there exists some $a_0\in R\smallsetminus (a_1, \cdots, a_k)$, simply add $a_0 x$ to the generators; and do the similar to ensure that the degree of polynomial associated with the corresponding coefficients is at least one. Since $R$ is Noetherian, it is finitely generated, i.e. the process above will terminate, which does not interfere with the condition that the ascending chain does not terminate.) 
    
    For $f_{k+1}$, we know that there exists a family $(c_j)_{j=1}^k$ s.t. $a_{k+1} = \sum\limits_{j=1}^k c_j a_j$ since $(a_1, \cdots, a_k)$ are generators. Then consider
    \[
        f = f_{k+1} - \sum\limits_{i=1}^k c_i x^{d_{k+1} - d_i} f_i
    \]
    which is a polynomial that is not in $I\smallsetminus (f_1, \cdots, f_n)$, which is a contradiction.
\end{proof}

\begin{corollary}
    By induction $R[x_1, \cdots, x_n]$ is also Noetherian if $R$ is Noetherian. Quotient and localization preserves the property that a ring is Noetherian.
\end{corollary}

\section{Euclidean Domain, PIDs and UFDs}

\begin{definition}[Principal Ideal Domain (PID)]
    Let $R$ be a integral domain. $R$ is a \textbf{Principal Ideal Domain (PID)} if every ideal in $R$ is principal.
\end{definition}

\begin{remark}
    If $R$ is a PID, then $R$ is Noetherian, as principal ideals are by definition finitely generated.
\end{remark}

\begin{proposition}\label{prop:PID prime is maximal}
    If $R$ is a PID, then every prime ideal in it is maximal.
\end{proposition}

\begin{proof}
    Prove by contradiction. Suppose that $I = (p)$ is a prime ideal that is not maximal. Then by Theorem \ref{thm:maximal ideal inclusion} there exists some maximal ideal $x\notin I$ s.t. $I \subseteq (x)$, i.e. there exists some $r\in R$ s.t. $p = xr$. Since $x\notin I, r\in I$. Write $r = pr'$ for $r'\in R$. Then $xr'=1$, i.e. $(x) = (1)$ which is a contradiction.
\end{proof}

\begin{definition}[Euclidean Domain]
    A \textbf{Euclidean Domain} is an integral domain $R$, for which there exists a function (norm) $N: R\smallsetminus\{0\} \to \Z_{\geq 0}$, s.t. $\forall a, b\in R, \neq 0$, there exists some $q, r\in R$ s.t. $a = bq + r$; and either $r = 0$, or $N(r) < N(b)$.
\end{definition}

\begin{proposition}\label{prop:Euclidean Domain is PID}
    A Euclidean Domain is a PID.
\end{proposition}

\begin{proof}
    Let $R$ be a euclidean domain. Since the domain of the norm is $\Z_{\geq 0}$, there exists some element $b$ s.t. $N(b)$ is minimal. Claim that $R = (b)$.

    This is indeed true, as there does not exist any $r$ s.t. $N(r) < N(b)$. Then apply the definition of a Euclidean Domain.
\end{proof}

\begin{definition}
    Let $a, b\in R\smallsetminus\{0\}$. Then $a$ is \textbf{associated with} $b$ (denoted $a \sim b$) if there exists some $u$ invertible, s.t. $a = ub$.
\end{definition}

\begin{remark}
    $a \sim b$ if and only if $(a) = (b)$.
\end{remark}

\begin{definition}[Greatest Common Divisor]
    Let $a, b\in R$ that are not both zero. The \textbf{Greatest Common Divisor} of $a$ and $b$ is an element in $R\smallsetminus\{0\}$ s.t.
    $d\mid a, d\mid b$; and for all $x\in R\smallsetminus\{0\}$, $x\mid a \wedge x\mid b \implies x \mid d$.
\end{definition}

\begin{proposition}\label{prop:gcd ideal}
    Let $R$ be a domain, and $d$ be the gcd of $a$ and $b$. If $(a, b) = (d)$, then $d = \gcd(a, b)$. 
\end{proposition}

\begin{proof}
    $d$ is a common divisor of $a$ and $b$ as $a, b\in (d)$. It is the greatest one as since $d\in (a, b)$, there exists some $\lambda, \mu\in R$ s.t. $\lambda a + \mu b = d$. Both sides should divide $d$, which implies that if there exists some $d' \mid a, d'\mid b$, then $d' \mid d$.
\end{proof}

\begin{definition}[Prime; Irreducible]
    Let $R$ be a domain, and $a$ a non-zero element. Then
    \begin{itemize}
        \item $a$ is a \textbf{prime} if $(a)$ is a prime ideal.
        \item $a$ is \textbf{irreducible} if for all $b_1, b_2\in R$ s.t. $a = b_1 b_2$, either $b_1$ is invertible or $b_2$ is invertible.
    \end{itemize}
\end{definition}

\begin{proposition}\label{prop:irreducible implies maximal}
    Let $R$ be a PID and $r\in R$ a non-zero element. Then $r$ is irreducible if and only if $(r)$ is a maximal ideal.
\end{proposition}

\begin{proof}
    Proceed by showing implication in two directions:
    \begin{itemize}
        \item[$\Rightarrow$:] Let $r$ be an irreducible element. Suppose that there exists an ideal $I$ s.t. $(r) \subsetneqq I \subsetneqq R$. Since $R$ is a PID, there exists some $a\in R$ s.t. $I = (a)$, which indicates that there exists some $x\in R$ s.t. $r = ax$. But since $r$ is irreducible, either $a$ is a unit, i.e. $I = R$, or $x$ is a unit, i.e. $I = (r)$. Both of which lead to a contradiction.
        \item[$\Leftarrow$:] Proceed by showing the contraposition. Suppose that $r$ is not irreducible, then there exists $p, q\in R$ which are not units s.t. $r = pq$. Then $(r) \subsetneqq (p) \subsetneqq R$ which implies that $(r)$ is not maximal.
    \end{itemize}
\end{proof}

\begin{proposition}\label{prop:prime is irreducible}
    If $a$ is prime, then $a$ is irreducible.
\end{proposition}

\begin{proof}
    Let $a$ be a prime. Suppose that there exists $b_1, b_2\in R$ s.t. $b_1 b_2 = a$. Then $b_1 b_2 \in (a)$. Without loss of generality assume $b_1\in (a)$, i.e. there exists some $r\in R$ s.t. $b_1 = ar$. This gives $arb_2 = a$, i.e. $b_2$ is invertible.
\end{proof}

\begin{remark}
    The converse is generally not true. Consider in $\Z[\sqrt{5}i]$ which is not a UFD. Then $(2)$ is not prime (as $2\cdot 3 = (1 + \sqrt{5}i)(1 - \sqrt{5}i)$) but $2$ is irreducible.
\end{remark}

\begin{definition}[Unique Factorization Domain (UFD)]
    A domain $R$ is a \textbf{Unique Factorization Domain (UFD)} if for all  nonzero $a\in R$ that is not invertible, there exists a decomposition $a = p_1\cdots p_r$ where $p_1, \cdots, p_r$ are irreducible. For all other families of irreducible elements $q_1, \cdots, q_r \in R$ s.t. $a = q_1\cdots q_r$, there exists a permutation $\varepsilon: [r+1] \to [r+1]$ s.t. $p_i \sim q_{\varepsilon(i)} \forall i$. 
\end{definition}

\begin{proposition}\label{prop:UFD irreducible is prime}
    Let $R$ be a UFD. Then every irreducible element $p\in R$ is prime.    
\end{proposition}

\begin{proof}
    Claim that $(p)$ is a prime ideal given that $p$ is irreducible. Since $p$ is irreducible and $R$ is UFD, for all $b_1 b_2\in (p)$, there exists some irreducible $q_i$s for $i\in I$ s.t. $b_1 b_2 = p\cdot \prod_{i\in I} q_i$. Since factorization unique, at least one of $b_1$ and $b_2$ admits a divisor $p$, which indicates that $(p)$ is a prime ideal. 
\end{proof}

\begin{proposition}\label{prop:irreducible is prime implies UFD}
    Let $R$ be a domain s.t. every irreducible element is prime. Then $R$ is a UFD.
\end{proposition}

\begin{proof}
    It suffices to prove that factorization is unique up to permutation and multiplication by units. Suppose that $p_i$s and $q_i$s are two irreducible decomposition of $a$, i.e. $a = p_1\cdots p_r = q_1 \cdots q_s$. Then either
    \begin{itemize}
        \item $r = 0$. Then $a$ is a unit, which indicates that $s = 0$.
        \item $r\neq 0$. Then $s\neq 0$. Since $p_i$ is prime for all $i$, there exists some $q_j$ s.t. $p_i \mid q_j$. this implies that $r\leq s$. Then consider $q_i$s as prime, which implies $s\leq r$ and therefore $s = r$. Further since $p_i$s and $q_i$s are irreducible, for $p_i \mid q_j$ this implies $q_j = p_i u$ for $u$ a unit. 
    \end{itemize}
    This verifies the definition of a UFD.
\end{proof}

\begin{proposition}\label{prop:Noeth implies decomposition}
    Let $R$ be a Noetherian ring. Then every element $a\in R$ attains an irreducible decomposition $a = p_1\cdots p_r$ with $p_i$ irreducible for all $i$. 
\end{proposition}

\begin{proof}
    This is simply a re-formalization of the fact that Noetherian rings are finitely generated. Consider the following cases:
    \begin{itemize}
        \item $a$ is irreducible. Then the factorization process is done.
        \item $a = b_1 b_2$ where $b_1$ and $b_2$ are both not units. Then consider separately $b_1$ and $b_2$ with this process. This process is sure to terminate at some point as otherwise this gives an ideal of infinite generators.
    \end{itemize}
\end{proof}

\begin{remark}
    Noetherian rings are generally not UFDs. A simple example is $\Z[\sqrt{5}i]$, the Gaussian Integers.
\end{remark}

\begin{theorem}\label{thm:PID is UFD}
    Every PID is a UFD.
\end{theorem}

\begin{proof}
    Since principal ideals are finitely generated, all PIDs are Noetherian. By proposition \ref{prop:Noeth implies decomposition} there exists a decomposition; and by proposition \ref{prop:irreducible implies maximal} and \ref{prop:PID prime is maximal} irreducible elements are prime. By proposition \ref{prop:irreducible is prime implies UFD} it is a UFD.
\end{proof}

\begin{example}
    An example where a ring is a UFD but not a PID (where prime ideals are not maximal) is $\Z[x]$, with the ideal $(2,x)$ which is not principal. $(x)$ is prime, but not maximal.
\end{example}

The following proves the theorem:

\begin{theorem}\label{thm:R UFD implies R[x] UFD}
    Let $R$ be a UFD, then $R[x]$ is also a UFD.
\end{theorem}

\begin{definition}[Primitive; Content]
    Let $f\in R[x]$ a nonzero polynomial. Then
    \begin{itemize}
        \item The \textbf{content} of $f$, denoted as $c(f)$ is the greatest common divisor of the coefficient of its terms.
        \item $f\in R[x]$ is \textbf{primitive} if its content is a unit.
    \end{itemize}
\end{definition}

\begin{lemma}\label{lem:criterion for polynomial irreducible}
    Let $R$ be a UFD. Define $K := \mathrm{Frac}(R)$, i.e. $K = S^{-1}R$ for $S := R\smallsetminus \{0\}$. A nonzero element $f\in R[x]$ is irreducible if and only if either of the following holds:
    \begin{itemize}
        \item $\deg f = 0$, and $f$ is irreducible in $R$.
        \item $\deg f \geq 1$, $f$ is primitive and is irreducible in $K[x]$. 
    \end{itemize}
\end{lemma}

\begin{proof}
    Consider the following two cases:
    \begin{itemize}
        \item $\deg f = 0$. Since $R\subseteq R[x]$, $f$ irreducible in $R[x]$ implies that it is irreducible in $R$. For the converse, notice that $R$ is a domain, where the degree of product of two polynomials is at the sum of the degree of the two polynomials, indicating that $f\in\R[x]$ could only attain degree 0 factors. The fact that $f$ is irreducible in $R$ finishes the proof.
        \item $\deg f \geq 1$. Consider the two directions:
            \begin{itemize}
                \item[$\Rightarrow$:] Suppose that $f$ is irreducible in $R[x]$. Notice that for all $g\in K[x]$, $c(g)^{-1} g\in R[x]$. Proceed by showing a contradiction. Suppose that there exists $f_1, f_2\in K[x]$ of degree at least one s.t. $f = f_1 f_2$ (i.e. $f$ is not irreducible in $K[x]$). Then
                $$
                    f = (c(f_1)^{-1} f_1) (c(f_2)^{-1} f_2) c(f_1) c(f_2)
                $$
                where the four operands for multiplication are all in $R$. Since $f$ is irreducible in $R$, either $(c(f_1)^{-1} f_1)$ or $(c(f_2)^{-1} f_2)$ is a unit, which contradicts the hypothesis that $\deg f_1 \geq 1 \wedge \deg f_2 \geq 1$.
                \item[$\Leftarrow$:] Proceed by showing that the contraposition is true. Suppose that $f = f_1 f_2$ where $f_1, f_2$ are both not units, in $R$. Then $f = f_1 f_2 \in K[x]$ which is also not irreducible.
            \end{itemize}
    \end{itemize}
\end{proof}

\begin{lemma}\label{lem:K[x] PID}
    Let $K$ be a field. Then $K[x]$ is a PID.
\end{lemma}

\begin{proof}
    Let $I$ be an ideal in $K[x]$. Define $k := \{ \deg f \mid f\in I \}$. Such $k$ indeed exists as the degree has a lower bound $0$; and $k$ could take only finitely many values with some element $f_0\in I$ fixed; namely $\llbracket 0, \deg f_0 \rrbracket$. Claim that $I = (x^k)$.

    Either $k = 0$, where $I = (1)$; or $k \neq 0$, where for all $f = \sum\limits_{i|d_i\geq d} c_i x^{d_i}$ $\sum\limits_{i} c_i x^{d_i - d} \in K[x]$. 
\end{proof}

\begin{proof}[Proof of Theorem \ref{thm:R UFD implies R[x] UFD}]
    Define $K = S^{-1}R$ for $S = R \smallsetminus\{0\}$. From lemma \ref{lem:K[x] PID} we know $K[x]$ is a PID, which is therefore a UFD. The general strategy is to transform the whole problem into $K[x]$ using lemma \ref{lem:criterion for polynomial irreducible}, and use the fact that $K[x]$ is a UFD, with elements differ only by a factor in $R$ (which is also a UFD) from those in $R[x]$.

    It suffices to show that the decomposition exists and is unique:
    \begin{itemize}
        \item \emph{Existence.} Decompose $f$ in $R[x]$ $f = c(f) g$ s.t. $g$ is primitive. Then $c(g) = u$ where $u$ is some unit in $R$. Applying the inclusion map gives $g\in K[x]$, where it could be decomposed into $g = g_1\cdots g_n$ where $g_i$s are irreducible. Denote $g_i = c(g_i) h_i$, which gives $g = \prod_{i=1}^n c(g_i) h_i = c(g) \prod_{i=1}^n h_1 = u\prod_{i=1}^n h_1$. Since $c(f) \in R$ which is a UFD, there exists a decomposition $c(f) = f_1\cdots f_n$. This gives an irreducible decomposition $f = f_1\cdots f_n h_1\cdots h_n$.
        \item \emph{Uniqueness.} This follows from the fact that both $f$ and $K[x]$ are UFDs, i.e. decomposition of $f\in R$ and $g\in K[x]$ are unique.
        
        (Alternatively one could prove that irreducible elements in $R[x]$ are also prime, which is essentially the same approach as the content is prime follows from the fact that $R$ is UFD; and the primitive is prime as $K[x]$ is a UFD).
    \end{itemize}
\end{proof}